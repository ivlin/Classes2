
\documentclass[11pt]{article}

\usepackage{graphicx} \usepackage{float} \usepackage{epstopdf}
\usepackage{xcolor}

\renewcommand{\baselinestretch}{1.2} \setlength{\topmargin}{-0.5in}
\setlength{\textwidth}{6.5in} \setlength{\oddsidemargin}{0.0in}
\setlength{\textheight}{9.1in}

\newlength{\pagewidth} \setlength{\pagewidth}{6.5in} \pagestyle{empty}

\def\pp{\par\noindent}

\begin{document}

\begin{flushright}

                 \textcopyright 2018 Michael A. Bender
\end{flushright}
\centerline{\bf CSE 350 -- Theory of Computation (Honors), Spring 2018}
\medskip
\centerline{Assignment 1}
\bigskip
\bigskip

You should write your problem set solutions using latex. For the Macintosh,
install Mactex and use Texshop. For Windows, install Miktex and use WinEdt or
some other editor. For linux, install texlive-full. For drawing figures, I
recommend using xfig, Ipe, inkscape, or the Mac tool Omnigraffle.

No late problem sets accepted.





\bigskip
\newcounter{problemctr}

\centerline{\bf Part A}



\addtocounter{problemctr}{1}
\bigskip
\noindent
$\underline{\rm Problem\ \theproblemctr}$\pp Let {\em T} be a transitive 
relation. Prove carefully or give a counterexample for the following 
statements.

\noindent
(1) The symmetric closure of {\em T} is transitive.

\noindent
(2) The reflexive closure of {\em T} is transitive.


\addtocounter{problemctr}{1}
\bigskip
\noindent
$\underline{\rm Problem\ \theproblemctr}$\pp Prove that in any group of more
than one person there are always two people that know the same number of people
within the group. Assume that knowing is symmetric.


\addtocounter{problemctr}{1}
\bigskip
\noindent
$\underline{\rm Problem\ \theproblemctr}$\pp For each of the following, state
whether the result can be finite, countably infinite, or uncountable (no proof
required). There may be more than one correct answer.

\noindent
(1) The union of two uncountable sets

\noindent
(2) The power set of a countably infinite set

\noindent
(3) The intersection of two uncountable sets

\noindent
(4) The intersection of countably many uncountable sets

\noindent
(5) The set difference of two uncountable sets

\noindent
(6) The union of uncountably many countably infinite sets


\bigskip
\bigskip
\centerline{\bf Part B}

\addtocounter{problemctr}{1}
\bigskip

\noindent
$\underline{\rm Problem\ \theproblemctr}$\pp Are the following statements true
or false? Justify your answers with a proof or counterexample.

\noindent
(1) If $L_1$ and $L_2$ are both infinite languages, $L_1L_2$ contains a string 
that is in neither $L_1$ nor $L_2$.

\noindent
(2) If $L$ is finite, then $|L^*| > |L|$.


\addtocounter{problemctr}{1}
\bigskip
\noindent
$\underline{\rm Problem\ \theproblemctr}$\pp Prove or disprove: the union of
countably many countably infinite sets is countable. 

%Aside: You may assume the Axiom of Choice (if you don't know what that means, don't worry).


\addtocounter{problemctr}{1}
\bigskip
\noindent
$\underline{\rm Problem\ \theproblemctr}$ (Extra Credit)\pp
\noindent
There are $N$ soldiers lined up, about to execute a hapless
prisoner. The lieutenant, who begins the firing process, is
located at one end of the line. The soldiers all want to fire
simultaneously. Unfortunately, the soldiers can only talk to their
immediate neighbors on the left and right. In addition, these
soldiers are not very intelligent, and they are just finite automata.
Thus, they only have {\em constant
memory\/}, independent of $N$. Notice that constant memory is very
little. It is not enough even to count up to $N$; this requires
$\log (n)$ bits. It is not enough to have a name, because unique
names also require $\log (n)$ bits.

\noindent
Devise a procedure that allows these soldiers to fire in unison. The
algorithm should have running time $O(n)$.

\noindent
Note that they move and interact at exactly
the same speed; i.e., they are synchronized.

\end{document}
