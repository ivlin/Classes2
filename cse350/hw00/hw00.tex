\documentclass[11pt]{article}
% \usepackage{times}
\usepackage{palatino}

\usepackage{enumitem}

\renewcommand{\baselinestretch}{1.2}
\setlength{\topmargin}{-0.5in}
\setlength{\textwidth}{6.5in}
\setlength{\oddsidemargin}{0.0in}
\setlength{\textheight}{9.1in}

\newlength{\pagewidth}
\setlength{\pagewidth}{6.5in}
\pagestyle{empty}

\def\pp{\par\noindent}

\special{papersize=8.5in,11in}


\begin{document}
\LARGE{Homework 0: Academic Dishonesty}\\
\normalsize
CSE350 - Dr. Michael Bender\\
Ivan Lin
\newcounter{problemctr}

\addtocounter{problemctr}{1}
\bigskip
\noindent
$\underline{\rm Problem\ \theproblemctr}$\pp
%
In this class I let students work together to solve problems, as long
as each student writes up her/her own solution and cites all
collaborators.  Why do I let students work together?  Why is this not
academically dishonest?

Students are allowed to work in groups because it promotes collaboration and problem solving in a team. This is an invaluable skill whether one hopes to go into research or industry, since there will inevitably be times where they need to productively work with others to come up with a solution. Part of this involves learning to make your own contributions to the group effort and properly citing the work done by others.

\addtocounter{problemctr}{1}
\bigskip
\noindent
$\underline{\rm Problem\ \theproblemctr}$\pp
%
Why is it important to your professional development to struggle with
a problem that you cannot solve quickly?  Why do I deliberately assign
homework that could take you days or weeks to solve?  What do you
expect to learn from this experience?

It is important that students learn to struggle with problems they can't quickly solve because it promotes determination, perseverence, and as Dr. Bender says, "grit." Regardless of how smart someone is, there are problems that won't be solved in a few minutes, hours, or even days and the ability to commit to something like that in the face of a lack of progress like that is a huge asset. Again, it's important regardless of where one ends up in life.

\addtocounter{problemctr}{1}
\bigskip
\noindent
$\underline{\rm Problem\ \theproblemctr}$\pp
%
Why is it academically dishonest to share your write-up?  Why is it
plagiarism to copy someone else's write-ups.  Give an example of
collaboration that is academically honest.  Give another example that
is academically dishonest.

Plagiarism is the act of submitting someone else's work or ideas as ones own. By giving out one's write up, it could either consciously or unconsciously influence other students' write up. It goes without saying that someone could copy the write up verbatim and attempt to submit it as their own, which would be dishonest to the teachers and unfair to their classmates. The sharer would also bear responsibility for putting the writeup out for another to use to their advantage. An honest example would be two students discussing potential solutions to a problem and through the discussion, they arrive at an answer. They then write it up separately while crediting each other for their contribution to obtaining the answer. A dishonest example would be one student completing the write up alone and showing it to another student who copies the answers without any citation or reference to the original author.

\addtocounter{problemctr}{1}
\bigskip
\noindent
$\underline{\rm Problem\ \theproblemctr}$\pp
%
Explain why it is academically dishonest to share your solution set
with another student.  Explain how you could get burned from just
sharing your writeup even if you do not copy yourself.

Sharing work is academic dishonesty because you know you are providing a method for the other student to pass off another's work as their own, to their own detriment and the detriment of all other students who honestly stuggled and worked on the problem. It is dishonest to those who sincerely worked for the answer and it will negatively impact the person who sees the answer. You are also assisting the student in fooling the teacher and TAs.

\addtocounter{problemctr}{1}
\bigskip
\noindent
$\underline{\rm Problem\ \theproblemctr}$\pp
%
Explain why copying (or approximating) solutions from the web (or
another source) is plagiarism, even if you cite your
source. In this course I ask that you \emph{not} to scour the
  web for solutions to the homework problems.  But even I did let you
  search the web to find solutions, this would \emph{still} be
  plagiarism.  Why?

If an answer is submitted from another source and you have not made a contribution to developing that answer, it is not your work to submit. You could easily have no understanding of the solution, but by handing it in, you are implying you arrived at the solution partially through your own efforts and understand the problem that you solved.

\addtocounter{problemctr}{1} \bigskip
\noindent
$\underline{\rm Problem\ \theproblemctr}$\pp
%
Why is it better for your grade to leave a question blank, rather than
search for answers on the web?  (Hint: calculate approximately how
much a homework problem is worth to your raw score versus an exam
question. You can include the risk-benefit analysis of getting
caught.)

By leaving a question blank, you are indicating to the grader of the homework or test that there is a problem or concept you are struggling with and need help on. In doing so, you can ask for clarification on that question and develop a real understanding of the solution for future situations, so there is a future benefit in doing so. If you look up the answer and get it right, the difference it makes in your grade is generally insignificant and you will have problems with similar type questions in the future due to lack of true understanding.

% MAB: I'm commenting out this problem, because I'd rather students
% not use the internet to help them solve problems at all.
%
%\addtocounter{problemctr}{1}
%\bigskip
%\noindent
%$\underline{\rm Problem\ \theproblemctr}$\pp
%Suppose that you are completely stuck about how to approach a problem.
%Give an example of using the internet to help that is academically honest.
%Give an example of using the internet that is academically dishonest.
%

\addtocounter{problemctr}{1}
\bigskip
\noindent
$\underline{\rm Problem\ \theproblemctr}$\pp
%
Imagine that you are employed at a major software company, say Google,
Facebook, or Microsoft, and commit code into a product that you copied
from a website.  Explain the potential risks to both you and the
company if this action is discovered by the owners of the code.

Committing copied code is incredibly dangerous since it opens the developer and their company to lawsuits. It damages the reputations of both, could result in the product being scrapped (or at least rewritten and reevaluated), and will likely cost the developer their job. It is the theft of intellectual property and could have costly repercussions in terms of money and time. Furthermore, if you don't understand the code you copied, it could lead to bugs or break the product.

\addtocounter{problemctr}{1}
\bigskip
\noindent
$\underline{\rm Problem\ \theproblemctr}$\pp
%
Why I have this problem set on academic honesty?

As you mentioned in class, many people have differing definitions of what counts as academic dishonest. By asking for this write up, this can serve as reference in the future in the case a student is caught being academically dishonest. They can't claim ignorance and the professor can simply check this assignment to see the students views on honesty (and whether those views have somehow changed).

\addtocounter{problemctr}{1}
\bigskip
\noindent
$\underline{\rm Problem\ \theproblemctr}$\pp
%
How much time did this writeup take you, including the time it took to
learn latex.

Having had Professor Bender last year for CSE150 where he expected all homeworks to be written up in \LaTeX - including an academic honesty astonishingly similar to this one - it took about 20 minutes. Since the questions are the same and my views are largely unchanged, the main ideas of all my responses are similar to my old answers. I paraphrased some of my old responses while expanding or removing some points.

\end{document}

