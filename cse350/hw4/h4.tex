
\documentclass[11pt]{article}

\usepackage{graphicx}
\usepackage{float}
\usepackage{epstopdf}
\usepackage{amssymb}
\usepackage{enumitem}

\setlist[enumerate]{topsep=0pt,itemsep=-1ex,partopsep=1ex,parsep=1ex,leftmargin=.5cm}

\renewcommand{\baselinestretch}{1.2}
\setlength{\topmargin}{-0.5in}
\setlength{\textwidth}{6.5in}
\setlength{\oddsidemargin}{0.0in}
\setlength{\textheight}{9.1in}

\newlength{\pagewidth}
\setlength{\pagewidth}{6.5in}
\pagestyle{empty}

\def\pp{\par\noindent}

\begin{document}

\centerline{\bf CSE 350 -- Theory of Computation (Honors), Spring 2018}
\medskip
\centerline{Assignment 4}

\bigskip
\bigskip

\newcounter{problemctr}

\addtocounter{problemctr}{1}

\noindent $\underline{\rm Problem\ \theproblemctr}$\pp
	Give a CFG that generates the language
	$$ A = \{ a^ib^jc^k \mid i,j,k \geq 0 \mbox{ and } i + k = j\} $$

\noindent
$S\rightarrow IK$\\
$I\rightarrow aIb$\\
$I\rightarrow e$\\
$K\rightarrow bKc$\\
$k\rightarrow e$

\bigskip
\addtocounter{problemctr}{1}

\noindent $\underline{\rm Problem\ \theproblemctr}$\pp
Show that the following language is not context free:
$$ L = \{ a^n \mid \mbox{ n is a perfect square}\} $$

Assume for the sake of contradiction that $L$ is context free.\\
According to the pumping lemma for context free langugaes, there exists a constant pumping length $p$ for which any string $s$ in the language can be written as $s=uvwxy$ where:
\begin{enumerate}
    \item $|vwx|\leq p$
    \item $|vx|\geq 1$
    \item $uv^nwx^ny \in L$ for all $n\geq0$
\end{enumerate}

We take string $a^{p^2}\in L$. By condition 1, there must be substrings $v=a^j,x=a^k,0<j+k\leq p$ that can be pumped $n$ times where $n\geq0$ and the string will be in the language.\\
This means $\forall n\geq0, a^{p^2+ni}\in L,0<i\leq p$. Suppose we pump the substrings once. $a^{p^2+n}\in L,0<i\leq p$.\\
However we know that the next longest word in the language is $a^{(p+1)^2}=a^{p^2+2p+1}$.\\
We arrive at a contradition since $2p+1>p\geq n \rightarrow 2p+1>n$, meaning $a^{p^2+ni}\not\in L$

\bigskip
\bigskip
\addtocounter{problemctr}{1}

\noindent $\underline{\rm Problem\ \theproblemctr}$\pp

\noindent Let $\Sigma=\{0,1\}$ and let B be the collection of strings that contain at least one 1 in their second half. In other words,\
$B=\{uv \mid u\in\Sigma^*, v\in\Sigma^ *1\Sigma^* \mbox{ and } |u|\geq|v|\}$.
Give a PDA that recognizes $B$.\\
$k: \{s,a,h\}$\\
$\Sigma: \{0,1\}$\\
$\delta:$
\begin{itemize}
    \item $(s,0,e)\rightarrow(s,0)$
    \item $(s,1,e)\rightarrow(s,1)$
    \item $(s,\Sigma,e)\rightarrow(a,e)$
    \item $(s,e,e)\rightarrow(a,e)$
    \item $(a,0,\Sigma)\rightarrow(a,e)$
    \item $(a,1,\Sigma)\rightarrow(h,e)$
    \item $(h,\Sigma,\Sigma)\rightarrow(h,e)$
\end{itemize}
$s:s$\\
$H:\{h\}$
\bigskip
\bigskip
\addtocounter{problemctr}{1}

\noindent $\underline{\rm Problem\ \theproblemctr}$\pp
\begin{enumerate}
	\item[a.]
	Show that the language $F=\{a^i b^j c^k \mid i,j,k \geqslant 0 \mbox{ and if } i=1 \mbox{ then } j=k\}$ satisfies the conditions of the pumping lemma for {\bf regular languages}.
	\item[b.]
	Show that despite this, $F$ is not regular.\\
\end{enumerate}

\noindent
a.\\
According to the pumping lemma for regular langugaes, there exists a constant pumping length $p$ for which any string $s$ in the language can be written as $s=xy^nz$ where:
\indent
\begin{enumerate}
    \item $|y|\geq 1$
    \item $|xy|\leq p$
    \item $xy^nz \in L$ for all $n\geq0$\\
\end{enumerate}

\noindent
Let us consider two complementary subsets of $L$.\\
\indent 1. $F^1=\{a^i b^j c^k \mid i,j,k\geq0, i\neq1\}$ - recognize that this is $a^*b^*c^*-ab^*c^*$\\
\indent 2. $F^2=\{a^i b^j c^k \mid i,j,k\geq0, i=1, j=k\}$\\
In the first case, we can pump any single character substring and it will remain in the language.
In the second case, if we let $v=a$, $a^1b^*c^*$ is in the language by definition while pumping it results in a string in $F^1$.\\

\noindent
b.\\
The Myhill Nerode theorem states that a language is regular if and only if there are a finite number of equivalence classes. This means if there are an infinite number of equivalence classes, then the language is not regular.\\
By the definition of equivalence, we know that two strings $x,y$ are in the same equivalence class if $\forall z, xz\in F \leftrightarrow yz\in F$.\\
We consider the set $G=\{a^i b^j c^k \mid i,j,k\geq0, i=1, j>k\}$. We see that $j=k$ for strings in $F$, so for any given $g=a b^j c^k, j>k$, there exists an equivalence class for strings where $z=c^{j-k}$ such that $gz\in F$.\\
This means there is an equivalence class for all unique solutions to $j-k, j>k\geq0$, which is infinite. By the Myhill Nerode theorem, we can conclude that there are an infinite number of equivalence classes for $F$.\\

---\\
Citations:\\
Johnny So - "discussion over brunch" - discussed potential ways to prove non regularity
\end{document}